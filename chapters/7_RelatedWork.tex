{\chapter{Related Work}
\label{chap:RelatedWork}


In (\ref{chap:Eval_6}), we presented the second and last part of the evaluation results for the experiments we conducted on the graph structures in order to evaluate the performance of the graph structures in query execution. In this chapter, we present an overview of other research that is related to the work we did in this thesis. This chapter is consisted of the following sections:

\begin{itemize}  

\item \textbf{Graph Structures Scalability:}\\
In section (\ref{sec:relatedWork-scalability}), we present other research that have been done to evaluate the scalability of graph data structures.

\item \textbf{Loading Techniques:}\\
In section (\ref{sec:relatedWork-loading}), we present other research that have been done to evaluate the performance of graph data loading techniques.

\item \textbf{Query Performance:}\\
In section (\ref{sec:relatedWork-query}), we present other research that have been done to evaluate the performance of query execution on data in the graph data structures.

\item \textbf{Graph Libraries:}\\
In section (\ref{sec:relatedWork-graphLibraries}), we present graph libraries that offer an alternative implementation to graph data structures and algorithms.

\end{itemize}


\section{Graph Structures Scalability}
\label{sec:relatedWork-scalability}

The scalability of a graph data structure is determined by its ability to process larger graphs. Scalability is measured by the amount of time and memory taken to load, modify, or perform computation on a graph with larger size. In a research done by \textit{(Wheatman and Xu)}, the authors presented a scalability evaluation for the loading, updating, and querying of a packed version of the compressed sparse row (CSR) structure against basic CSR, adjacency list, and other variants of adjacency list \cite{wheatmanpacked}. \textit{King et al.} have presented a scalability evaluation for a dynamic version of CSR for (GPU) processing \cite{King2016DynamicCSRA}.

\section{Loading Techniques}
\label{sec:relatedWork-loading}

\textit{Then et al.} have presented evaluation of a set of strategies that can be utilized for the parallel loading of graph data into graph structures. The authors have tested their proposed parallel loading strategies on an implemented version of the compressed sparse row (CSR) and map of neighbour list \cite{then2016evaluation}.

\section{Query Performance}
\label{sec:relatedWork-query}

\textit{Abadi et al.} have presented an evaluation for the performance of query execution on a set of graph properties structures that included the emerging schema structure as well as other graph properties structures \cite{abadi2007scalable}. \textit{Wheatman and Xu} as well as \textit{Blandford et al.} have presented an evaluation for the performance of query execution on a set of graph topology structures that included the compressed sparse row structure and the adjacency list structure among others \cite{wheatmanpacked,blandford2004experimental}.


\section{Graph Libraries}
\label{sec:relatedWork-graphLibraries}

In this thesis, we developed our own implementation of the graph data structures. Other graph libraries that offer an alternative implementation of graph data structures do exist. The \textit{Boost Graph Library (BGL)} is providing a standardized generic interface for building graph data structures and traversal algorithms for the stored graph data. The boost graph library is implemented using the \textit{C++} programming language and offers an implementation of a set of graph data structures (e.g. adjacency list and adjacency matrix) \cite{boostGraphLibrary2001}. The \textit{Stanford Network Analysis Platform (SNAP)} is a network analysis library that has a back-end written in \textit{C++} and an interface that is offered in \textit{python}. SNAP supports the representation of directed, undirected, and multi-graph types. Internally, SNAP is using an adjacency list in order to represent the graph topology \cite{leskovec2016snap}. \textit{NetworkX} is a graph library that is written in \textit{python} and offers data structures for storing graph data as well as algorithms for network analysis and traversal. NetworkX can operate over graphs of type directed, undirected, simple, pseudo, or multi-graph. NetworkX is offering three graph structures (edge list, adjacency matrix, or adjacency list) for the storage of graph topology \cite{SciPyProceedings_11}.


}